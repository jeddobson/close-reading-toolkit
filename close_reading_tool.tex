%
% chart for close reading toolkit
%

\documentclass[12pt]{article}

\usepackage[utf8]{inputenc}
\usepackage[table]{xcolor}
\usepackage{multirow,graphicx}
\usepackage[margin=.25in,landscape]{geometry}

\usepackage[T1]{fontenc}
\usepackage[urw-garamond]{mathdesign}

\renewcommand{\arraystretch}{1}
 
\begin{document}

\setlength{\tabcolsep}{8pt}

\arrayrulecolor[HTML]{000000}
\setlength{\arrayrulewidth}{.75mm}

\begin{center}
\Large{Ways of Reading: \\
A Toolkit for Close Reading of Literary Texts}
\end{center}

\begin{table}[ht]
\centering

\begin{tabular}{| m{2mm} *{5}{|m{1.75in}} |}
\hline
\rowcolor[gray]{0.8}
& 
Audience & 
Rhetorical Strategy & 
Organizational \newline Structure & 
Pronoun Use & 
Narration \\ \hline
\centering\parbox[t]{2mm}{\rotatebox[origin=c]{90}{Rhetorical Features}} &

Who is being addressed? What assumptions does the text make about the reader? &
Does the text position itself through the authority of the author (ethos), an intended emotional response (pathos), or with a logical argument (logos)? \newline &
What are the important organizational features? How does the text introduce and conclude itself? &
What pronouns are used? How do these gesture toward the author or the reader? &
What characterizes the narrative voice? Is it written the first or third
person? Retrospective or in the present? \\ 

\hline 
\rowcolor[gray]{0.8}
&
Diction (Word-Choice) &
Connotation \newline / Denotation &
Repetitions &
Similarities &
Contradictions \newline or Binarries \\ 
\hline

\centering\parbox[t]{2mm}{\rotatebox[origin=c]{90}{Textual Features}} &

Why were these specific words were selected and not others? &
What the implications or assumptions of the connotative meaning of specific words? &
What words, phrases, or images, repeat? &
Are there similarities of words or phrases within the text? What are the
meaning differences of these similarities? \newline &
What oppositional pairings of words or phrases do you see in the text? \\ 
\hline

\rowcolor[gray]{0.8}
& 
Genre &
Metaphors &
Allusions &
Figurative Tropes &
Rhythm \\ \hline

\centering\parbox[t]{2mm}{\rotatebox[origin=c]{90}{Literary Features}} &

Can you identify the genre of the text? Does it participate in multiple
genres?& 

Do you see a juxtaposition of two independent ideas or objects used to show
similarities between both? &

Are there possible references to other texts, people, stories, or myths not
explicitly represented within the text? \newline&

Are there examples of metonymy, synecdoche, irony, alliteration, litotes,
hyperbole, or personification?&

Does the text produce any sort of rhythm through the repetition of the same or
similar sounds? \\ \hline

\end{tabular}
\end{table}
\vspace*{1in}
\begin{center}
\small{MIT Licensed. Created by James E. Dobson (2016)}
\end{center}
\end{document}
