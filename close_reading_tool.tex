\documentclass[12pt]{article}
\usepackage[table]{xcolor}
%\usepackage{extsizes}
\usepackage{multirow,graphicx}
\usepackage[margin=.25in,landscape]{geometry}

\usepackage[T1]{fontenc}
\usepackage[urw-garamond]{mathdesign}

\begin{document}

\begin{center}
\Large{Ways of Reading: \\
A Toolkit for Close Reading of Literary Texts}
\end{center}

\begin{table}[ht]
\centering

\begin{tabular}{| p{2mm} | p{1.75in} | p{1.75in} | p{1.75in} | p{1.75in} | p{1.75in}|}
\hline
\rowcolor[gray]{0.8}
&
Audience & 
Rhetorical Strategy & 
Organizational Structure & 
Pronoun Use & 
Narration \\ \hline
\centering\parbox[t]{2mm}{\rotatebox[origin=c]{90}{Rhetorical Features}} &
Who is being addressed? What assumptions does the text make about the reader? &
Does the text position itself through the authority of the author (ethos), an intended emotional response (pathos), or with a logical argument (logos)? &
What are the important organizational features? How does the text introduce and conclude itself? &
What pronouns are used? How do these gesture toward the author or the reader? &
What characterizes the narrative voice? Is it written the first or third
person? Retrospective or in the present? \\ 

\hline 
\rowcolor[gray]{0.8}
&
Dicton (Word-Choice) &
Connotation / Denotation &
Repetitions &
Similarities &
Contradictions or Binarries \\ 
\hline

\centering\parbox[t]{2mm}{\rotatebox[origin=c]{90}{Rhetorical Features}} &
Why were these specific words were selected and not others? &
What the implications or assumptions of the connotative meaning of specific words? &
What words, phrases, or images, repeat? &
Are there similarities of words or phrases within the text? What are the
meaning differences of these similarities? &
What oppositional pairings of words or phrases do you see in the text? \\ 
\hline

%\multirow{2}{*}{\rotatebox[origin=c]{90}{Rhetorical Features}} &
%\cellcolor[gray]{0.8} Genre &
%\cellcolor[gray]{0.8} Metaphors &
%\cellcolor[gray]{0.8} Allusions &
%\cellcolor[gray]{0.8} Figurative Tropes &
%\cellcolor[gray]{0.8} Rhythm \\ \hline
%
%Can you identify the genre of the text? Does it participate in multiple
%genres?& 
%
%Do you see a juxtaposition of two independent ideas or objects used to show
%similarities between both? &
%
%Are there possible references to other texts, people, stories, or myths not
%explicitly represented within the text?&

%Are there examples of metonymy, synecdoche, irony, alliteration, litotes,
%hyperbole, or personification?&

%Does the text produce any sort of rhythm through the repetition of the same or
%similar sounds? \\ \hline

\end{tabular}
\end{table}
\end{document}
